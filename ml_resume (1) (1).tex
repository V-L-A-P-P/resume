\documentclass{resume-faangpath}

% --- Поддержка кириллицы ---
\usepackage{fontspec}
\setmainfont{Roboto}

\usepackage{titlesec}
\usepackage[normalem]{ulem}
\usepackage{hyperref}

\titleformat*{\section}{\Large\bfseries}


\titlespacing{\section}{0pt}{20pt}{8pt}


% Увеличиваем шрифт названий проектов + добавляем дату над названием
\renewcommand{\resumeProjectHeading}[2]{%
  \vspace{1mm}
  {\small \textcolor{darkgray}{#2}}\par
  {\large\textbf{#1}}%
  \vspace{1mm}
}

% Полное переопределение \name для поддержки двух строк
\renewcommand{\name}[2]{%
  \begin{center}
      {\Huge \bfseries #1}\\[2mm] % имя
      {\large \textbf{#2}}       % должность
  \end{center}
}

\definecolor{darkgray}{RGB}{105, 105, 105}


\hypersetup{
  colorlinks=true,
  urlcolor=darkgray
}

\newcommand{\projectsep}{%
  {\color{gray}\noindent\rule{\linewidth}{0.4pt}}%
  \vspace{2mm}
}

\usepackage{enumitem}
\setlist[itemize]{topsep=2pt, itemsep=4pt, parsep=0pt, partopsep=0pt}



\begin{document}

%-------------------- HEADER --------------------
\name
  {Апухтин Владислав Борисович}
  {ML Engineer / Data Scientist (Intern/Junior)}


\contact{+7-915-062-2993}{\href{mailto:apukhtin00@inbox.ru}{apukhtin00@inbox.ru}}{Москва}{\href{https://t.me/kyl0q}{Tg: @kyl0q}}{\href{https://github.com/V-L-A-P-P}{github.com/V-L-A-P-P}}

%-------------------- SUMMARY --------------------
\begin{summary}
Студент направления «Прикладное машинное обучение». Разрабатываю высоконадёжные ML-решения: от подготовки данных и построения моделей до развёртывания сервисов (FastAPI + Docker). Реализовал проекты в предсказательной аналитике, NLP, RAG и рекомендательных системах. Участвую в олимпиадах, хакатонах, публикую статьи по ML. Нацелен на развитие и решение прикладных ML-задач в команде профессионалов.
\end{summary}


%-------------------- EXPERIENCE --------------------
\section{Опыт работы}
\resumeSubheading
{Algorithm Engineer (Project-based)}{Сентябрь 2025 -- н.в.}
{RWB — оптимизация вещания рекламных кампаний (в рамках KIM Project) }{Москва}

\vspace{-0mm}

\resumeItemListStart
  \resumeItem{Разрабатываю алгоритм оптимизации рекламного вещания с учётом бизнес-ограничений, овербукинга и равномерной ротации показов.}
  \resumeItem{Проектирую алгоритмическое решение на базе \textbf{комбинаторной оптимизации} для индустриального продукта.}
\resumeItemListEnd

\vspace{-0mm}
%-------------------- PROJECTS --------------------
\section{Проекты}
\resumeProjectHeading
{\href{https://github.com/V-L-A-P-P/Bankruptcy_Prediction_System}{\uline{Bankruptcy Prediction System}} — production-ready ML-сервис для оценки риска банкротства}
{Май 2025 -- Июнь 2025}

{\textcolor{gray}{
Python | Pandas | Scikit-learn | CatBoost | Optuna | SHAP | FastAPI | PyTest | Docker
}}


\resumeItemListStart
  \resumeItem{Собрал и структурировал крупный финансовый датасет (≈700k компаний, 110 признаков) с выраженным дисбалансом классов; провёл глубокий \textbf{EDA} и \textbf{feature engineering} (125 доменных и производных признаков).}
  \resumeItem{Построил модели и настроил гиперпараметры через Optuna с учётом сильного дисбаланса (1/250); финальная модель достигла \textbf{weighted accuracy = 0.87}.}
  \resumeItem{Окалибровал вероятности с помощью Isotonic Regression и внедрил интерпретацию модели через \textbf{SHAP}.}
  \resumeItem{Реализовал \textbf{полный ML-пайплайн} обучения и инференса, тестирование (\textbf{PyTest}) и деплой сервиса (\textbf{FastAPI}, \textbf{Docker}).}
\resumeItemListEnd



\projectsep




\resumeProjectHeading
{\href{https://github.com/V-L-A-P-P/AURA}{\uline{AURA}} — интеллектуальная RAG-система поиска и генерации ответов по финтех-корпусу}
{Август 2025 -- Сентябрь 2025}

{\textcolor{gray}{
Python | FAISS (HNSW) | Sentence-Transformers | Cross-Encoder | Graph Ranking | LLM | NetworkX | FastAPI | Docker
}}


\resumeItemListStart
\resumeItem{Реализовал \textbf{гибридный retrieval} (TF-IDF + FAISS + двухэтапный cross-encoder reranking на \textbf{transformers}), обеспечив прирост качества относительно базового RAG.}
\resumeItem{Внедрил \textbf{LLM-based query rewriting} на базе instruction-tuned моделей для повышения полноты и релевантности поиска.}
\resumeItem{Построил пайплайн подготовки данных с несколькими стратегиями чанкинга (базовый, семантический, рекурсивный) и предварительным препроцессингом документов.}
\resumeItem{Разработал \textbf{граф связей между чанками} и интегрировал его в retrieval-этап для graph-augmented расширения кандидатов и reranking; реализовал \textbf{end-to-end пайплайн} с тестами и деплоем.}
\resumeItemListEnd


\projectsep


\resumeProjectHeading
{\href{https://github.com/V-L-A-P-P/Movie_Recommendation_System}{\uline{Movie Recommendation System}} — ML-модель персональных рекомендаций фильмов}
{Июнь 2025 -- Июль 2025}



{\textcolor{gray}{
Python | Pandas | NumPy | SciPy | Scikit-learn | Implicit ALS | Collaborative Filtering | Ranking Metrics
}}

\resumeItemListStart
  \resumeItem{Разработал персонализированную рекомендательную систему фильмов с использованием \textbf{User-Based CF}, \textbf{Item-Based CF} и \textbf{confidence-weighted implicit ALS}.}
  \resumeItem{Реализовал \textbf{cold-start handling} и \textbf{explainable рекомендации}: явное объяснение выбора фильмов через вклад похожих ранее просмотренных items (item–item similarity).}
  \resumeItem{Спроектировал \textbf{единый evaluation pipeline} и провёл сравнение моделей по \textbf{Recall@10}, \textbf{NDCG@10} и \textbf{MAP@10}}
\resumeItemListEnd




%-------------------- SKILLS --------------------
\section{Навыки}

\resumeSkillListStart

  \resumeSkill{Языки и инструменты}
    {Python, SQL, Git, Bash, Linux}

  \resumeSkill{ML и анализ данных}
    {Pandas, NumPy, Scikit-learn, CatBoost, XGBoost, LightGBM, Optuna, SHAP}

  \resumeSkill{Deep Learning}
    {PyTorch, Transformers, RAG, HuggingFace}

  \resumeSkill{MLOps и сервисы}
    {FastAPI, Docker, PyTest}

  \resumeSkill{Математика}
    {линейная алгебра, матанализ, теория вероятностей, математическая статистика}

\resumeSkillListEnd

%-------------------- EDUCATION --------------------
\section{Образование}
\resumeSubheading
{Финансовый университет при Правительстве РФ}{2024 -- 2028}
{Факультет информационных технологий и анализа больших данных}{Москва}
{Направление/профиль: прикладная математика и информатика/прикладное машинное обучение}

\textbf{Курсы:}
\begin{itemize}[nosep]
    \item \textit{Практический Machine Learning (Сертификат с отличием)} — AI Education \hfill 2025
    \item \textit{Продвинутые методы машинного обучения (Сертификат с отличием)} — AI Education \hfill 2025
    \item \textit{Практический Deep Learning (Сертификат с отличием)} — AI Education \hfill 2025
    \item \textit{Введение в Data Science и машинное обучение (Сертификат с отличием)} — Karpov Courses \hfill 2025
\end{itemize}


%-------------------- EXTRA --------------------
\section{Дополнительно}
\resumeItemListStart
  \resumeItem{Активно участвую в хакатонах и соревнованиях по анализу данных, помогаю в организации хакатонов}
  \resumeItem{Пишу статьи и регулярно изучаю новые подходы в ML/Deep Learning.}

  \resumeItem{Языки: Русский — Native, Английский — B2, Немецкий — A2}
  \resumeItem{Личные качества: ответственность, аналитическое мышление, стрессоустойчивость, коммуникабельность}

  
\resumeItemListEnd

\end{document}